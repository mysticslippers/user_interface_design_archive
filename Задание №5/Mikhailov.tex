\documentclass[12pt]{report}


\usepackage[T1,T2A]{fontenc}
\usepackage[utf8]{inputenc}
\usepackage[english, russian]{babel}
\usepackage{geometry}
\geometry{left=25mm, right=20mm, top=20mm, bottom=20mm}

\usepackage{graphicx}
\usepackage{float}
\usepackage{booktabs}
\usepackage{longtable}
\usepackage{array}
\usepackage{ragged2e}
\usepackage{hyperref}
\usepackage{xcolor}
\usepackage{tabularx}
\usepackage{float}

\usepackage{minted}
\setminted{
	fontsize=\small,
	breaklines=true,
	frame=single,
	bgcolor=gray!5
}

\usepackage{amsmath, amssymb}

\setlength{\parindent}{1.25cm}
\setlength{\parskip}{0.5em}
\renewcommand{\arraystretch}{1.2}
\sloppy

\begin{document}
	
	\begin{titlepage}
		\begin{center}
			\large{Федеральное государственное автономное образовательное учреждение высшего образования <<Национальный исследовательский университет ИТМО>>}
		\end{center}
		
		\vspace{15em}
		
		\begin{center}
			\huge{\textbf{Курсовая работа}} \\
			\large{По дисциплине <<Проектирование Пользовательских Интерфейсов>>} \\
			\large{Задание №5} \\
		\end{center}
		
		\vspace{2em}
		
		\begin{flushright}
			\textit{\large{Выполнили:}} \\
			\large{Студент группы P3306} \\
			\large{Михайлов Дмитрий} \\
			\large{Андреевич} \\
			\large{Студент группы P3317} \\
			\large{Мищенко Роман} \\
			\large{Андреевич} \\
			\textit{\large{Преподаватель:}} \\
			\large{Балканский Андрей} \\
			\large{Александрович}
		\end{flushright}
		
		\vspace{2cm}
		
		\begin{figure}[h]
			\centering
			\includegraphics[width=0.5\linewidth]{image.png}
		\end{figure}
		
		\begin{center}
			Санкт-Петербург \\
			2025 год
		\end{center}
	\end{titlepage}
	
	\tableofcontents
	\newpage
	
	\addcontentsline{toc}{section}{Задание}
	\section*{Задание}
	Провести юзабилити-тестирование библиотечного веб-сервиса (версия для читателя и версия для библиотекаря), подготовить отчет по шаблону, выявить проблемы, возникающие у пользователей при использовании сервиса, и сформировать рекомендации по улучшению интерфейса. В ходе тестирования осуществлялась запись экрана респондента и фиксировались его действия при выполнении заданий.
	
	\addcontentsline{toc}{section}{Цель исследования}
	\section*{Цель исследования}
	Оценить удобство выполнения ключевых пользовательских сценариев в библиотечном веб-сервисе: поиск и выбор книги, регистрация/вход и взаимодействие с карточкой книги (версия для читателя), а также управление каталогом и обработка действий пользователей (версия для библиотекаря).
	
	\addcontentsline{toc}{section}{Задачи}
	\section*{Задачи}
	\begin{enumerate}
		\item Проверить, могут ли пользователи без подсказок выполнить ключевые сценарии использования сервиса.
		\item Выявить проблемы интерфейса, которые приводят к ошибкам, замедляют выполнение задач или вызывают затруднения.
		\item Собрать комментарии и субъективные оценки респондентов по удобству сервиса.
		\item Сформировать рекомендации по улучшению интерфейса на основе результатов тестирования.
	\end{enumerate}
	\newpage
	
	\addcontentsline{toc}{section}{Гипотеза}
	\section*{Гипотеза}
	Пользователи смогут успешно выполнить основные сценарии работы с сервисом (поиск и выбор книг, регистрация/вход, выполнение целевого действия с книгой) без существенных затруднений, а библиотекарь сможет управлять каталогом и статусами операций без критических ошибок. Ожидается, что выявленные проблемы будут носить локальный характер и устраняться за счёт улучшения фильтров, структуры форм и подсказок интерфейса.
	
	\addcontentsline{toc}{section}{Параметры поиска респондентов}
	\section*{Параметры поиска респондентов}
	В тестировании участвовали 7 респондентов, разделённых на две группы:
	\begin{itemize}
		\item 5 респондентов тестировали версию сайта \textbf{читателя};
		\item 2 респондента тестировали версию сайта \textbf{библиотекаря}.
	\end{itemize}
	
	Критерии подбора респондентов:
	\begin{itemize}
		\item базовые навыки работы с веб-сайтами (поиск информации, использование форм, навигация по страницам);
		\item опыт использования сервисов с каталогами и поиском (интернет-магазины, библиотеки, маркетплейсы и т.п.);
		\item для группы библиотекаря — понимание типовых задач администрирования каталога (добавление/редактирование данных, изменение статусов).
	\end{itemize}
	
	\addcontentsline{toc}{section}{План тестирования}
	\section*{План тестирования}
	
	\subsection*{Вступление}
	\begin{enumerate}
		\item Кратко объяснить цель исследования: тестируется интерфейс сервиса, а не навыки пользователя.
		\item Попросить респондента проговаривать свои мысли вслух.
		\item Предупредить о записи экрана и действий пользователя в ходе выполнения заданий.
		\item Уточнить общий опыт: как часто респондент пользуется онлайн-каталогами/поиском/фильтрами.
	\end{enumerate}
	
	\subsection*{Задания}
	
	\textbf{Для версии читателя (5 респондентов):}
	\begin{enumerate}
		\item Найти книгу по названию или автору через строку поиска.
		\item Найти книгу с использованием фильтров (включая фильтр по году).
		\item Зарегистрироваться/войти в аккаунт и оценить понятность формы регистрации.
		\item Открыть карточку выбранной книги и определить её доступность (в наличии/занята и т.п.).
		\item Выполнить целевое действие с книгой (например, забронировать или добавить в список/избранное --- в зависимости от функциональности сервиса).
	\end{enumerate}
	
	\textbf{Для версии библиотекаря (2 респондента):}
	\begin{enumerate}
		\item Найти книгу в каталоге административной версии сервиса.
		\item Добавить новую книгу в каталог, заполнив обязательные поля.
		\item Отредактировать данные существующей книги (например, год, жанр, количество экземпляров).
		\item Обработать действие пользователя (например, бронь/выдачу), изменив статус операции.
	\end{enumerate}
	
	\subsection*{Общие вопросы после тестирования}
	\begin{enumerate}
		\item Что показалось самым удобным в сервисе?
		\item Что было непонятно или вызвало затруднения?
		\item На каком шаге вы потратили больше всего времени и почему?
		\item Чего не хватило в поиске и фильтрах?
		\item Что бы вы изменили в первую очередь?
		\item Оцените удобство сервиса по шкале от 1 до 10 и кратко объясните оценку.
	\end{enumerate}
	\newpage
	
	\addcontentsline{toc}{section}{Условные обозначения}
	\section*{Условные обозначения}
	\textbf{Частота} --- минимальный процент пользователей, которые могут сталкиваться с проблемой.
	\textbf{Степень критичности} определяется по частоте:
	\begin{itemize}
		\item в диапазоне от 50\% --- \textbf{высокая} степень критичности;
		\item от 20\% до 50\% --- \textbf{средняя} степень критичности;
		\item до 20\% --- \textbf{низкая} степень критичности.
	\end{itemize}
	
	\addcontentsline{toc}{section}{Таблица выявленных проблем}
	\section*{Таблица выявленных проблем}
	
	\renewcommand{\arraystretch}{1.15}
	\setlength{\tabcolsep}{4pt}
	
	\begin{table}[H]
		\footnotesize
		\centering
		
		\begin{tabularx}{\textwidth}{|
				>{\raggedright\arraybackslash}p{3.2cm}|
				>{\raggedright\arraybackslash}X|
				>{\centering\arraybackslash}p{2.2cm}|
				>{\centering\arraybackslash}p{2.4cm}|
			}
			\hline
			\textbf{Раздел приложения} & \textbf{Проблема} & \textbf{Частота} & \textbf{Критичность} \\
			\hline
			
			Поиск / фильтры (читатель) &
			Недостаточно фильтров для поиска нужной книги (сложно сузить выдачу до релевантного результата). &
			5/5 (100\%) & Высокая \\
			\hline
			
			Поиск / фильтры (читатель) &
			Фильтр по году реализован неудобно: нужен выпадающий список или чек-боксы по годам / диапазону лет. &
			3/5 (60\%) & Высокая \\
			\hline
			
			Регистрация (читатель) &
			При регистрации не нужна роль «Читатель»: выбор роли лишний и сбивает с толку. &
			4/5 (80\%)  & Высокая \\
			\hline
			
			Поиск (читатель) &
			Нет сортировки результатов (например, по году/популярности/новизне), из-за чего сложнее выбрать подходящую книгу. &
			2/5 (40\%)  & Средняя \\
			\hline
			
			Поиск / навигация (читатель) &
			Фильтры и/или поисковый запрос сбрасываются при переходе на карточку книги и возврате назад (приходится настраивать заново). &
			2/5 (40\%)  & Средняя \\
			\hline
			
			Добавление/редакт книги (библиотекарь) &
			В форме добавления/редактирования книги неочевидны обязательные поля и формат ввода (не хватает подсказок и валидации). &
			2/2 (100\%) & Высокая \\
			\hline
			
		\end{tabularx}
	\end{table}
	\newpage
	
	\addcontentsline{toc}{section}{Рекомендации}
	\section*{Рекомендации}
	
	\subsection*{1. Недостаточно фильтров для поиска нужной книги (5/5, 100\%)}
	\begin{itemize}
		\item Добавить расширенные фильтры: жанр, автор, язык, издательство, наличие (в наличии/занята), библиотека/филиал (если применимо).
		\item Сделать возможность выбора нескольких значений фильтра (например, несколько жанров).
		\item Показывать количество результатов рядом с фильтрами и/или мгновенно обновлять выдачу при изменении фильтров.
	\end{itemize}
	
	\subsection*{2. Неудобный фильтр по году (3/5, 60\%)}
	\begin{itemize}
		\item Реализовать выбор года через выпадающий список или чек-боксы по популярным годам/десятилетиям.
		\item Добавить диапазон лет (``от''--``до'') с валидацией и подсказкой формата.
	\end{itemize}
	
	\subsection*{3. Лишняя роль «Читатель» при регистрации (4/5, 80\%)}
	\begin{itemize}
		\item Убрать выбор роли ``Читатель'' из формы регистрации (по умолчанию регистрируется читатель).
		\item Роль ``Библиотекарь'' выдавать отдельно: через приглашение/админом/служебную регистрацию.
	\end{itemize}
	
	\subsection*{4. Нет сортировки результатов (2/5, 40\%)}
	\begin{itemize}
		\item Добавить сортировку: по релевантности, по году (возр./убыв.), по популярности/рейтингу (если есть), по новизне.
		\item Сохранять выбранную сортировку при переходе в карточку книги и возврате назад.
	\end{itemize}
	
	\subsection*{5. Сбрасываются фильтры/запрос при возврате назад (2/5, 40\%)}
	\begin{itemize}
		\item Сохранять состояние фильтров и строку поиска при переходе в карточку книги и возврате в список.
		\item Добавить кнопку ``Сбросить фильтры'' (явный сброс вместо скрытого).
	\end{itemize}
	
	\subsection*{6. Админка: неясны обязательные поля и формат ввода (2/2, 100\%)}
	\begin{itemize}
		\item Пометить обязательные поля (*), добавить подсказки и примеры заполнения.
		\item Добавить валидацию ввода (например, год --- только число, допустимый диапазон).
		\item После сохранения показывать понятное подтверждение (``Книга добавлена/обновлена'') и выделять ошибки рядом с полем.
	\end{itemize}
	
	\addcontentsline{toc}{section}{Комментарии респондентов}
	\section*{Комментарии респондентов}
	
	\begin{itemize}
		\item ``Не хватает фильтров --- сложно найти нужную книгу, слишком много лишнего в выдаче.''
		\item ``По году хотелось бы выбирать из списка или диапазоном, а не искать вручную.''
		\item ``Зачем выбирать роль читателя? Я же и так просто регистрируюсь как пользователь.''
		\item ``Было бы удобнее сортировать результаты, например по году или новизне.''
		\item ``Когда возвращаюсь назад, всё сбрасывается --- приходится настраивать заново.''
		\item ``В админке непонятно, какие поля обязательны и в каком формате вводить.''
	\end{itemize}
	
	\addcontentsline{toc}{section}{Вывод}
	\section*{Вывод}
	Проведено юзабилити-тестирование библиотечного веб-сервиса: 5 респондентов тестировали версию читателя и 2 респондента --- версию библиотекаря. В ходе тестирования фиксировались действия пользователей и выполнялась запись экрана.
	
	Гипотеза подтвердилась частично: основные сценарии в целом выполнялись, однако выявлены проблемы, которые заметно ухудшают пользовательский опыт. Наиболее критичными оказались недостаточное количество фильтров для поиска, неудобная работа с фильтром по году, а также проблемы формы регистрации. Для версии библиотекаря критичной оказалась недостаточная понятность формы добавления/редактирования книги (обязательные поля и формат ввода).
	
	Рекомендуется в первую очередь улучшить систему фильтрации и сохранение параметров поиска, упростить регистрацию (убрать лишний выбор роли), а также доработать формы админки за счёт подсказок и валидации. Эти изменения должны снизить количество ошибок и ускорить выполнение ключевых сценариев.
	
\end{document}
