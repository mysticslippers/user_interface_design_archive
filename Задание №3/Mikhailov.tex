\documentclass[12pt]{report}


\usepackage[T1,T2A]{fontenc}
\usepackage[utf8]{inputenc}
\usepackage[english, russian]{babel}
\usepackage{geometry}
\geometry{left=25mm, right=20mm, top=20mm, bottom=20mm}

\usepackage{graphicx}
\usepackage{float}
\usepackage{booktabs}
\usepackage{longtable}
\usepackage{array}
\usepackage{ragged2e}
\usepackage{hyperref}
\usepackage{xcolor}

\usepackage{minted}
\setminted{
	fontsize=\small,
	breaklines=true,
	frame=single,
	bgcolor=gray!5
}

\usepackage{amsmath, amssymb}

\setlength{\parindent}{1.25cm}
\setlength{\parskip}{0.5em}
\renewcommand{\arraystretch}{1.2}
\sloppy

\begin{document}
	
	\begin{titlepage}
		\begin{center}
			\large{Федеральное государственное автономное образовательное учреждение высшего образования <<Национальный исследовательский университет ИТМО>>}
		\end{center}
		
		\vspace{15em}
		
		\begin{center}
			\huge{\textbf{Курсовая работа}} \\
			\large{По дисциплине <<Проектирование Пользовательских Интерфейсов>>} \\
			\large{Задание №3} \\
		\end{center}
		
		\vspace{2em}
		
		\begin{flushright}
			\textit{\large{Выполнили:}} \\
			\large{Студент группы P3306} \\
			\large{Михайлов Дмитрий} \\
			\large{Андреевич} \\
			\large{Студент группы P3317} \\
			\large{Мищенко Роман} \\
			\large{Андреевич} \\
			\textit{\large{Преподаватель:}} \\
			\large{Балканский Андрей} \\
			\large{Александрович}
		\end{flushright}
		
		\vspace{2cm}
		
		\begin{figure}[h]
			\centering
			\includegraphics[width=0.5\linewidth]{image.png}
		\end{figure}
		
		\begin{center}
			Санкт-Петербург \\
			2025 год
		\end{center}
	\end{titlepage}
	
	\tableofcontents
	\newpage
	
	\addcontentsline{toc}{section}{Задание}
	\section*{Задание}
	Разработать информационную архитектуру (структуру, информационные элементы, названия разделов). Не нужно указывать, где какая кнопка/картинка(это будет видно после выполнения 4 лабораторной), только информационная структура из разделов и сущностей.
	\begin{enumerate}
		\item Декомпозиция - выделите информационные сущности(страницы, товары, пользователи и т.д.) и опишите их свойства(какую информацию содержит в себе сущность: id, название, описание, изображение, размер и т.д.) \\
		
		\item Рассмотрите, как взаимодействуют между собой сущности, являются они вложенными(как каталог->категория->товар) или равными, составьте структуру информационной системы. Для навигации по продукту старайтесь использовать до 4 основных разделов (максимум 7), а остальные делать вложенными в один из основных. \\
		
		\item Проверьте, понятные ли названия у компонентов вашей системы. Для этого обратитесь к человеку из вашей целевой аудитории, кратко опишите ему идею вашего продукта, и попросите его описать, какая информация находиться под тем или иным элементом системы. Следите за длиной названия(если более 3 слов, то возможно стоит разнести этот компонент на два) \\
		
		\item Схематизация - составьте схему информационной архитектуры в формате текста/блок-схемы \\
		
		\item Загрузите схему информационной системы на яндекс-диск. \\
	\end{enumerate}
	\newpage
	
	\addcontentsline{toc}{section}{ER-model}
	\section*{ER-model}
	
	\begin{figure}[h]
		\centering
		\includegraphics[width=0.8\textwidth]{ER-model.png}
		\caption{ER-модель на основе описаний предметной области и прецедентов из предыдущего этапа.}
		\label{fig:ER-model}
	\end{figure}
	\newpage
	
	\addcontentsline{toc}{section}{Схема информационной архитектуры}
	\section*{Схема информационной архитектуры}
	
	\renewcommand{\arraystretch}{1.2}
	
	\begin{longtable}{|p{0.35\textwidth}|p{0.6\textwidth}|}
		\hline
		\textbf{Раздел / Уровень} & \textbf{Описание / Содержание} \\ \hline
		\endfirsthead
		\hline
		\textbf{Раздел / Уровень} & \textbf{Описание / Содержание} \\ \hline
		\endhead
		
		\textbf{Главная страница} &
		Стартовая страница системы. Предлагает авторизацию пользователя (вход по логину и паролю). После успешного входа перенаправление в раздел «Личный кабинет». \\ \hline
		
		\hspace*{1em}Авторизация &
		Форма входа: логин, пароль, выбор роли (если предусмотрено автоматическое определение — скрыто). Возможность восстановления пароля. \\ \hline
		
		\textbf{Личный кабинет} &
		Главный раздел после входа в систему. Отображает индивидуальный интерфейс в зависимости от роли пользователя. \\ \hline
		
		\hspace*{1em}Роль: \textbf{Читатель} &
		\RaggedRight
		• \textbf{Читательский билет} — информация о статусе, номере, активных бронированиях.
		• \textbf{Мои бронирования} — список активных и завершённых бронирований.
		• \textbf{Мои штрафы} — начисленные и оплаченные штрафы. \\ \hline
		
		\hspace*{1em}Роль: \textbf{Библиотекарь} &
		\RaggedRight
		• \textbf{Список бронирований пользователей} — просмотр и управление заявками на бронирование.
		• \textbf{Управление выдачами} — оформление и возврат книг пользователями. \\ \hline
		
		\hspace*{1em}Роль: \textbf{Администратор} &
		\RaggedRight
		• \textbf{Управление библиотекой} — редактирование информации о библиотеке, добавление/удаление материалов.
		• \textbf{Управление пользователями} — просмотр и редактирование данных сотрудников и читателей. \\ \hline
		
		\textbf{Библиотеки} &
		Раздел, доступный всем ролям (в том числе неавторизованным пользователям при необходимости). Содержит информацию о библиотеках, их сотрудниках и фондах. \\ \hline
		
		\hspace*{1em}Список библиотек &
		Просмотр всех библиотек, подключённых к системе. Для каждой — название, адрес, контактная информация. \\ \hline
		
		\hspace*{2em}Информация о библиотеке &
		Детальная карточка библиотеки: адрес, статус, количество сотрудников, часы работы. \\ \hline
		
		\hspace*{2em}Список сотрудников &
		Отображает библиотекарей и администраторов, закреплённых за данной библиотекой. \\ \hline
		
		\hspace*{1em}Каталог материалов &
		Доступный каталог всех изданий конкретной библиотеки. Возможен поиск, фильтрация, просмотр карточки книги (название, автор, год, язык, наличие). \\ \hline
		
		\end{longtable}
	
\end{document}
